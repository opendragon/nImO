\ProvidesFile{MpMexamples.tex}[v1.0.0]
\appendixStart{\textitcorr{Examples}}
The examples that are part of the \mplusm{} set of source files are provided to support
the development of custom applications.
There are example clients, services, adapters as well as examples of the more specialized
Input, Output and Filter services.\\

All the example applications respond to a `HUP' signal and will exit gracefully when one
is sent to them.
\secondaryStart{Example~Services}
The set of example services include a pair of simple services
[\examplesNameR{Services}{m+mEchoService} and
\examplesNameR{Services}{m+mRandomNumberService}], a service with client context
[\examplesNameR{Services}{m+mRunningSumService}] as well as two input services
[\examplesNameR{Services}{m+mPlaybackFromJSONInputService} and
\examplesNameR{Services}{m+mRandomBurstInputService}], two output services
[\examplesNameR{Services}{m+mRecordAsJSONOutputService} and\\
\examplesNameR{Services}{m+mRecordIntegersOutputService}] and two filter services
[\examplesNameR{Services}{m+mAbsorberFilterService} and\\
\examplesNameR{Services}{m+mTruncateFloatFilterService}].\\

Note that services normally have no direct interface.
That is, they are `faceless' applications and perform their operations without user
interaction.
\tertiaryStart{\examplesNameP{Services}{m+mAbsorberFilterService}}
The \examplesNameX{Services}{m+mAbsorberFilterService} application is a Filter
service, accepting (and ignoring) any input data, while counting the number of messages
and bytes received, which are (optionally) sent out the `stats' outlet.
The application responds to the standard Filter service requests and can be used as a
standalone data transformer, without the need for a client connection.\\

The \requestsNameR{\inputOutput}{InputOutput}{configuration} request has no arguments and
returns the integer value for the `sample interval'.\\

The \requestsNameR{\inputOutput}{InputOutput}{configure} request has one argument
\longDash{} an integer value for the `sample interval'.
This value will be applied when the service is started or restarted.\\

The \requestsNameR{\inputOutput}{InputOutput}{restartStreams} request stops and then
starts the input stream.\\

The \requestsNameR{\inputOutput}{InputOutput}{startStreams} request initiates listening
on the input stream.\\

The \requestsNameR{\inputOutput}{InputOutput}{stopStreams} request terminates listening
on the input stream.\\

Note that the application will exit if the \serviceNameR[\RS]{RegistryService} is not
running.
\condPage{}
The application has one optional argument \longDash{} the sample interval in seconds.
If the sample interval is greater than zero, this specifies how often the messages per
second (and bytes per second) are reported.
The statistics are output to the console from which the application was run as well as to
the `stats' outlet.
If the sample interval is not greater than zero, each incoming message is reported as it
is received, along with the total number of bytes received since the service was started
or restarted.
The reporting is only done to the console from which the application was run.
\insertAppParameters
\insertTagDescription{Absorber Filter}
\insertFilterServiceComment\\

Note that the sample interval can also be set via commands, if the application is
running from a terminal.\\

\insertStandardServiceCommands
\condPage
If the service is selected for execution from within the \emph{\MMMU} application, the
following dialog will be presented:
\objScaledDiagram{mpm_images/launchAbsorberFilterService}%
{launchServiceAbsorber}{Launch options for the \emph{Absorber Filter} service}{0.8}

\insertTagAndEndpointDescription{mpm_images/runningAbsorberFilterService}%
{serviceRunningAbsorber}{The \emph{\MMMU} entity for the \emph{Absorber Filter} service}%
{1.0}
\condPage{}
If the \textbf{Configure the service} menu item has been selected, the following will
appear:
\objScaledDiagram{mpm_images/configureAbsorberFilterService}%
{configureServiceAbsorber}{Configuration window for the \emph{Absorber Filter}
service}{1.0}
\tertiaryEnd{\examplesNameE{Services}{m+mAbsorberFilterService}}
\tertiaryStart{\examplesNameP{Services}{m+mEchoService}}
The \examplesNameX{Services}{m+mEchoService} application responds to
\requestsNameR{Miscellaneous}{Miscellaneous}{echo} requests from the
\examplesNameR{Clients}{m+mEchoClient} application; it simply returns any arguments
sent with the request.\\

Note that the application will exit if the \serviceNameR[\RS]{RegistryService} is not
running.\\

\insertAutoAppParameters
\insertTagDescription{Echo}
The matching client application does not directly use the endpoint name to establish a
connection to the service.

If the service is selected for execution from within the \emph{\MMMU} application, the
following dialog will be presented:
\objScaledDiagram{mpm_images/launchEchoService}%
{launchServiceEcho}{Launch options for the \emph{Echo} service}{0.8}

\insertTagAndEndpointDescription{mpm_images/runningEchoService}%
{serviceRunningEcho}{The \emph{\MMMU} entity for the \emph{Echo} service}{1.0}
\tertiaryEnd{\examplesNameE{Services}{m+mEchoService}}
\tertiaryStart{\examplesNameP{Services}{m+mPlaybackFromJSONInputService}}
The \examplesNameX{Services}{m+mPlaybackFromJSONInputService} application is an Input
service, generating a stream of \yarp{} messages that have been recorded by the
\examplesNameR{Services}{m+mRecordAsJSONOutputService} application in a file.
The application responds to the standard Input service requests and can be used as a
standalone data generator, without the need for a client connection.\\

The \requestsNameR{\inputOutput}{InputOutput}{configuration} request has no arguments and
returns the floating\longDash{}point value for the `playback ratio', the
floating\longDash{}point value for the initial delay and an integer value for the
`loop flag'.
The `playback ratio' is a scaling factor for the interval between the sending of each
message from the file; a value of \asBoldCode{1} indicates no scaling while, for example,
a value of \asBoldCode{2} indicates that the messages will be sent half as often as they
were recorded.
A value of \asBoldCode{0} for the `loop flag' indicates no looping \longDash{} the data
contained in the file will be sent once, after the initial delay \longDash{} and a value
of \asBoldCode{1} indicates that the data contained in the file will be sent
continuously.\\

The \requestsNameR{\inputOutput}{InputOutput}{configure} request has three arguments
\longDash{} a floating\longDash{}point value for the `playback ratio', a
floating\longDash{}point value for the initial delay and an integer value for the
`loop flag'.
These values will be applied when the playback service is started or restarted.\\

The \requestsNameR{\inputOutput}{InputOutput}{restartStreams} request stops and then
starts the playback engine, which restarts the output \yarp{} network connection.\\

The \requestsNameR{\inputOutput}{InputOutput}{startStreams} request starts a playback
engine for the output stream, using the configured `playback ratio', `initial delay' and
`loop flag'.
Once started, the playback engine will send the data contained in the specified file via
the output \yarp{} network connection.\\

The \requestsNameR{\inputOutput}{InputOutput}{stopStreams} request stops the playback
engine, which stops the output \yarp{} network connection.\\

Note that the application will exit if the \serviceNameR[\RS]{RegistryService} is not
running.\\

The application has one required argument \longDash{} the path to the \json{} file
containing the messages to be used.
The application has three optional arguments \longDash{} the playback ratio, the initial
delay in seconds and a flag indicating whether the data is to be sent continuously or just
once.
\insertAppParameters
\condPage
\insertTagDescription{Playback From JSON Input}
\insertInputServiceComment\\

The other parameters provide the playback ratio, initial delay and loop flag; if not
specified, a playback ratio of \asBoldCode{1}, an initial delay of zero seconds and a loop
flag of \asBoldCode{1} \longDash{} indicating continuous sending of the data \longDash{}
will be used.
Note that the playback ratio, initial delay and loop flag can also be set via commands,
if the application is running from a terminal.\\

\insertStandardServiceCommands
\condPage{}
If the service is selected for execution from within the \emph{\MMMU} application, the
following dialog will be presented:
\objScaledDiagram{mpm_images/launchPlaybackFromJSONInputService}%
{launchServicePlaybackFromJSON}{Launch options for the \emph{Playback From JSON Input}
service}{0.8}

Note that the input file path is required.
\insertTagAndEndpointDescription{mpm_images/runningPlaybackFromJSONInputService}%
{serviceRunningPlaybackFromJSON}{The \emph{\MMMU} entity for the \emph{Playback From JSON
Input} service}{1.0}
\condPage{}
If the \textbf{Configure the service} menu item has been selected, the following will
appear:
\objScaledDiagram{mpm_images/configurePlaybackFromJSONInputService}%
{configureServicePlaybackFromJSON}{Configuration window for the \emph{Playback From JSON
Input} service}{1.0}

The input file name forms part of the names of the channels and is added to the title bar
of the service, as though it were a `tag'.\\

If a `tag' value is supplied, it is merged with the input file name, as shown here:
\objScaledDiagram{mpm_images/playbackJSONWithTag}%
{playbackJSONWithTag}{The \emph{\MMMU} entity for the \emph{Playback From JSON Input}
service with a tag}{1.0}

If an `endpoint' value is specified, the input file name forms part of the title bar but
does not affect the channel names:
\objScaledDiagram{mpm_images/playbackJSONWithEndpoint}%
{playbackJSONWithEndpoint}{The \emph{\MMMU} entity for the \emph{Playback From JSON Input}
service with an endpoint}{1.0}
\condPage{}
If both a `tag' value and an `endpoint' value are provided, the `endpoint' value
determines the channel names while the `tag' value is combined with the input file name
as part of the title bar:
\objScaledDiagram{mpm_images/playbackJSONWithEndpointAndTag}%
{playbackJSONWithEndpointAndTag}{The \emph{\MMMU} entity for the \emph{Playback From JSON
Input} service with an endpoint and tag}{1.0}

If the `mod' option is used, the `tag' option is primarily affected.
For example, if a `tag' value is supplied along a tag modifier of 1 byte, the following
will be displayed for the service:
\objScaledDiagram{mpm_images/playbackJSONWithTagM1}%
{playbackJSONWithTagM1}{The \emph{\MMMU} entity for the \emph{Playback From JSON Input}
service with a tag and one byte tag modifier}{1.0}

If an `endpoint' value is specified along with a tag modifier of 1 byte, the input file
name forms part of the title bar but does not affect the channel names and the tag
modifier is used to form the tag:
\objScaledDiagram{mpm_images/playbackJSONWithEndpointM1}%
{playbackJSONWithEndpointM1}{The \emph{\MMMU} entity for the \emph{Playback From JSON
Input} service with an endpoint and one byte tag modifier}{1.0}

If both a `tag' value and an `endpoint' value are provided along with a tag modifier of 1
byte, the `endpoint' value determines the channel names while the `tag' value is combined
with the input file name and the tag modifier as part of the title bar:
\objScaledDiagram{mpm_images/playbackJSONWithEndpointAndTagM1}%
{playbackJSONWithEndpointAndTagM1}{The \emph{\MMMU} entity for the \emph{Playback From
JSON Input} service with an endpoint, tag and one byte tag modifier}{1.0}
\condPage{}
If only a tag modifier is set \longDash{} for example a 2 byte tag modifier \longDash{}
then the tag modifier is combined with the input file name and affects both the channel
names and the title bar:
\objScaledDiagram{mpm_images/playbackJSONWithM2}%
{playbackJSONWithM2}{The \emph{\MMMU} entity for the \emph{Playback From JSON Input}
service with a two byte tag modifier}{1.0}
\tertiaryEnd{\examplesNameE{Services}{m+mPlaybackFromJSONInputService}}
\tertiaryStart{\examplesNameP{Services}{m+mRandomBurstInputService}}
The \examplesNameX{Services}{m+mRandomBurstInputService} application is an Input service,
generating a stream of random floating\longDash{}point values, with a given `burst size'
\longDash{} the number of values sent in a group, and a `burst period' \longDash{} the
number of seconds between groups.
The application responds to the standard Input service requests and can be used as a
standalone data generator, without the need for a client connection.\\

The \requestsNameR{\inputOutput}{InputOutput}{configuration} request has no arguments and
returns the floating\longDash{}point value for the `burst period' and an integer value for
the `burst size'.\\

The \requestsNameR{\inputOutput}{InputOutput}{configure} request has two arguments
\longDash{} a floating\longDash{}point value for the `burst period' and an integer value
for the `burst size'.
These values will be applied when the random number generator is started or restarted.\\

The \requestsNameR{\inputOutput}{InputOutput}{restartStreams} request stops and then
starts the random number generator, which restarts the output \yarp{} network
connection.\\

The \requestsNameR{\inputOutput}{InputOutput}{startStreams} request starts a random
number generator for the output stream, using the configured `burst period' and
`burst size'.
Once started, the random number generator will send groups of random
floating\longDash{}point values via the output \yarp{} network connection.\\

The \requestsNameR{\inputOutput}{InputOutput}{stopStreams} request stops the random
number generator, which stops the output \yarp{} network connection.\\

Note that the application will exit if the \serviceNameR[\RS]{RegistryService} is not
running.\\

The application has two optional arguments \longDash{} the burst period, in seconds and
the number of random values to generate in each burst.
\insertAppParameters
\insertTagDescription{Random Burst Input}
\insertInputServiceComment\\

The other parameters provide the initial burst period and size; if not specified, a burst
period of 1 second and burst size of 1 will be used.
Note that the burst period and size can also be set via commands, if the application is
running from a terminal.\\

\insertStandardServiceCommands
\condPage{}
If the service is selected for execution from within the \emph{\MMMU} application, the
following dialog will be presented:
\objScaledDiagram{mpm_images/launchRandomBurstInputService}%
{launchServiceRandomBurst}{Launch options for the \emph{Random Burst Input} service}{0.8}

\insertTagAndEndpointDescription{mpm_images/runningRandomBurstInputService}%
{serviceRunningRandomBurst}{The \emph{\MMMU} entity for the \emph{Random Burst Input}
service}{1.0}
\condPage{}
If the \textbf{Configure the service} menu item has been selected, the following will
appear:
\objScaledDiagram{mpm_images/configureRandomBurstInputService}%
{configureServiceRandomBurst}{Configuration window for the \emph{Random Burst Input}
service}{1.0}
\tertiaryEnd{\examplesNameE{Services}{m+mRandomBurstInputService}}
\condPage
\tertiaryStart{\examplesNameP{Services}{m+mRandomNumberService}}
The \examplesNameX{Services}{m+mRandomNumberService} application responds to
\requestsNameR{Examples}{Examples}{random} requests from the
\examplesNameR{Clients}{m+mRandomNumberClient} application; it returns a group of random
numbers to the client.\\

Note that the application will exit if the \serviceNameR[\RS]{RegistryService} is not
running.\\

\insertAutoAppParameters
\insertTagDescription{Random Number Input}
The matching client application does not directly use the endpoint name to establish a
connection to the service.\\

The \requestsNameD{Examples}{Examples}{random} request has an argument of the
count of floating\longDash{}point random numbers to be returned as a response to the
request.
\requestsNameE{Examples}{Examples}{random}%
\condPage
If the service is selected for execution from within the \emph{\MMMU} application, the
following dialog will be presented:
\objScaledDiagram{mpm_images/launchRandomNumberService}%
{launchServiceRandomNumber}{Launch options for the \emph{Random Number} service}{0.8}

\insertTagAndEndpointDescription{mpm_images/runningRandomNumberService}%
{serviceRunningRandomNumber}{The \emph{\MMMU} entity for the \emph{Random Number}
service}{1.0}
\tertiaryEnd{\examplesNameE{Services}{m+mRandomNumberService}}
\condPage
\tertiaryStart{\examplesNameP{Services}{m+mRecordAsJSONOutputService}}
The \examplesNameX{Services}{m+mRecordAsJSONOutputService} application is an Output
service, recording a stream of \yarp{} values as \json{} structures to an external file.
The application responds to the standard Output service requests and can be used as a
standalone data generator, without the need for a client connection.\\

The \requestsNameR{\inputOutput}{InputOutput}{configuration} request has no arguments and
returns the file\longDash{}system path to use for the output file.\\

The \requestsNameR{\inputOutput}{InputOutput}{configure} request has a single argument,
the file\longDash{}system path to use for the output file.
The path will be used when the input stream is started or restarted.\\

The \requestsNameR{\inputOutput}{InputOutput}{restartStreams} request stops and then
starts the input stream.\\

The \requestsNameR{\inputOutput}{InputOutput}{startStreams} request opens a file to be
used for output, using the configured output file path.\\

The \requestsNameR{\inputOutput}{InputOutput}{stopStreams} request closes the output
file that is being used.\\

Note that the application will exit if the \serviceNameR[\RS]{RegistryService} is not
running.\\

The application has one optional argument \longDash{} the output file path to be used.
\insertAppParameters
\insertTagDescription{Record as JSON Output}
\insertOutputServiceComment\\

The output file path parameter provides the initial output file path; if not specified, a
random path in the system temporary directory will be used.
For \win, the temporary directory being used is ``\textbackslash{}tmp'' while, for \osx{},
it will be ``/tmp''.
Note that the output file path can also be set via commands, if the application is
running from a terminal.\\

\insertStandardServiceCommands

If the service is selected for execution from within the \emph{\MMMU} application, the
following dialog will be presented:
\objScaledDiagram{mpm_images/launchRecordAsJSONOutputService}%
{launchServiceRecordAsJSON}{Launch options for the \emph{Record As JSON Output} service}%
{0.8}
\condPage
\insertTagAndEndpointDescription{mpm_images/runningRecordAsJSONOutputService}%
{serviceRunningRecordAsJSON}{The \emph{\MMMU} entity for the \emph{Record As JSON Output}
service}{1.0}

If the \textbf{Configure the service} menu item has been selected, the following will
appear:
\objScaledDiagram{mpm_images/configureRecordAsJSONOutputService}%
{configureServiceRecordAsJSON}{Configuration window for the \emph{Record As JSON Output}
service}{1.0}

The output file name forms part of the names of the channels and is added to the title bar
of the service, as though it were a `tag'.\\

If a `tag' value is supplied, it is merged with the output file name, as shown here:
\objScaledDiagram{mpm_images/recordJSONWithTag}%
{recordJSONWithTag}{The \emph{\MMMU} entity for the \emph{Record As JSON Output} service
with a tag}{1.0}

If an `endpoint' value is specified, the output file name forms part of the title bar but
does not affect the channel names:
\objScaledDiagram{mpm_images/recordJSONWithEndpoint}%
{recordJSONWithEndpoint}{The \emph{\MMMU} entity for the \emph{Record As JSON Output}
service with an endpoint}{1.0}
\condPage{}
If both a `tag' value and an `endpoint' value are provided, the `endpoint' value
determines the channel names while the `tag' value is combined with the output file name
as part of the title bar:
\objScaledDiagram{mpm_images/recordJSONWithEndpointAndTag}%
{recordJSONWithEndpointAndTag}{The \emph{\MMMU} entity for the \emph{Record As JSON
Output} service with an endpoint and tag}{1.0}

If the `mod' option is used, the `tag' option is primarily affected.
For example, if a `tag' value is supplied along a tag modifier of 1 byte, the following
will be displayed for the service:
\objScaledDiagram{mpm_images/recordJSONWithTagM1}%
{recordJSONWithTagM1}{The \emph{\MMMU} entity for the \emph{Record As JSON Output} service
with a tag and one byte tag modifier}{1.0}

If an `endpoint' value is specified along with a tag modifier of 1 byte, the script file
name forms part of the title bar but does not affect the channel names and the tag
modifier is used to form the tag:
\objScaledDiagram{mpm_images/recordJSONWithEndpointM1}%
{recordJSONWithEndpointM1}{The \emph{\MMMU} entity for the \emph{Record As JSON Output}
service with an endpoint and one byte tag modifier}{1.0}

If both a `tag' value and an `endpoint' value are provided along with a tag modifier of 1
byte, the `endpoint' value determines the channel names while the `tag' value is combined
with the script file name and the tag modifier as part of the title bar:
\objScaledDiagram{mpm_images/recordJSONWithEndpointAndTagM1}%
{recordJSONWithEndpointAndTagM1}{The \emph{\MMMU} entity for the \emph{Record As JSON
Output} service with an endpoint, tag and one byte tag modifier}{1.0}
\condPage{}
If only a tag modifier is set \longDash{} for example a 2 byte tag modifier \longDash{}
then the tag modifier is combined with the script file name and affects both the channel
names and the title bar:
\objScaledDiagram{mpm_images/recordJSONWithM2}%
{recordJSONWithM2}{The \emph{\MMMU} entity for the \emph{Record As JSON Output} service
with a two byte tag modifier}{1.0}
\tertiaryEnd{\examplesNameE{Services}{m+mRecordAsJSONOutputService}}
\tertiaryStart{\examplesNameP{Services}{m+mRecordIntegersOutputService}}
The \examplesNameX{Services}{m+mRecordIntegersOutputService} application is an Output
service, recording a stream of integer values to an external file.
The application responds to the standard Output service requests and can be used as a
standalone data generator, without the need for a client connection.\\

The \requestsNameR{\inputOutput}{InputOutput}{configuration} request has no arguments and
returns the file\longDash{}system path to use for the output file.\\

The \requestsNameR{\inputOutput}{InputOutput}{configure} request has a single argument,
the file\longDash{}system path to use for the output file.
The path will be used when the input stream is started or restarted.\\

The \requestsNameR{\inputOutput}{InputOutput}{restartStreams} request stops and then
starts the input stream.\\

The \requestsNameR{\inputOutput}{InputOutput}{startStreams} request opens a file to be
used for output, using the configured output file path.\\

The \requestsNameR{\inputOutput}{InputOutput}{stopStreams} request closes the output
file that is being used.\\

Note that the application will exit if the \serviceNameR[\RS]{RegistryService} is not
running.\\

The application has one optional argument \longDash{} the output file path to be used.
\insertAppParameters
\insertTagDescription{Record Integers Output}
\insertOutputServiceComment\\

The output file path parameter provides the initial output file path; if not specified, a
random path in the system temporary directory will be used.
For \win, the temporary directory being used is ``\textbackslash{}tmp'' while, for \osx{},
it will be ``/tmp''.
Note that the output file path can also be set via commands, if the application is
running from a terminal.\\

\insertStandardServiceCommands
\condPage{}
If the service is selected for execution from within the \emph{\MMMU} application, the
following dialog will be presented:
\objScaledDiagram{mpm_images/launchRecordIntegersOutputService}%
{launchServiceRecordIntegers}{Launch options for the \emph{Record Integers Output}
service}{0.8}

\insertTagAndEndpointDescription{mpm_images/runningRecordIntegersOutputService}%
{serviceRunningRecordIntegers}{The \emph{\MMMU} entity for the \emph{Record Integers
Output} service}{1.0}
\condPage{}
If the \textbf{Configure the service} menu item has been selected, the following will
appear:
\objScaledDiagram{mpm_images/configureRecordIntegersOutputService}%
{configureServiceRecordIntegers}{Configuration window for the \emph{Record Integers
Output} service}{1.0}

The output file name forms part of the names of the channels and is added to the title bar
of the service, as though it were a `tag'.\\

If a `tag' value is supplied, it is merged with the output file name, as shown here:
\objScaledDiagram{mpm_images/recordIntegersWithTag}%
{recordIntegersWithTag}{The \emph{\MMMU} entity for the \emph{Record Integers Output}
service with a tag}{1.0}

If an `endpoint' value is specified, the output file name forms part of the title bar but
does not affect the channel names:
\objScaledDiagram{mpm_images/recordIntegersWithEndpoint}%
{recordIntegersWithEndpoint}{The \emph{\MMMU} entity for the \emph{Record Integers Output}
service with an endpoint}{1.0}

If both a `tag' value and an `endpoint' value are provided, the `endpoint' value
determines the channel names while the `tag' value is combined with the output file name
as part of the title bar:
\objScaledDiagram{mpm_images/recordIntegersWithEndpointAndTag}%
{recordIntegersWithEndpointAndTag}{The \emph{\MMMU} entity for the \emph{Record Integers
Output} service with an endpoint and tag}{1.0}
\condPage{}
If the `mod' option is used, the `tag' option is primarily affected.
For example, if a `tag' value is supplied along a tag modifier of 1 byte, the following
will be displayed for the service:
\objScaledDiagram{mpm_images/recordIntegersWithTagM1}%
{recordIntegersWithTagM1}{The \emph{\MMMU} entity for the \emph{Record Integers Output}
service with a tag and one byte tag modifier}{1.0}

If an `endpoint' value is specified along with a tag modifier of 1 byte, the script file
name forms part of the title bar but does not affect the channel names and the tag
modifier is used to form the tag:
\objScaledDiagram{mpm_images/recordIntegersWithEndpointM1}%
{recordIntegersWithEndpointM1}{The \emph{\MMMU} entity for the \emph{Record Integers
Output} service with an endpoint and one byte tag modifier}{1.0}

If both a `tag' value and an `endpoint' value are provided along with a tag modifier of 1
byte, the `endpoint' value determines the channel names while the `tag' value is combined
with the script file name and the tag modifier as part of the title bar:
\objScaledDiagram{mpm_images/recordIntegersWithEndpointAndTagM1}%
{recordIntegersWithEndpointAndTagM1}{The \emph{\MMMU} entity for the \emph{Record Integers
Output} service with an endpoint, tag and one byte tag modifier}{1.0}

If only a tag modifier is set \longDash{} for example a 2 byte tag modifier \longDash{}
then the tag modifier is combined with the script file name and affects both the channel
names and the title bar:
\objScaledDiagram{mpm_images/recordIntegersWithM2}%
{recordIntegersWithM2}{The \emph{\MMMU} entity for the \emph{Record Integers Output}
service with a two byte tag modifier}{1.0}
\tertiaryEnd{\examplesNameE{Services}{m+mRecordIntegersOutputService}}
\condPage
\tertiaryStart{\examplesNameP{Services}{m+mRunningSumService}}
The \examplesNameX{Services}{m+mRunningSumService} application responds to
requests from the \examplesNameR{Clients}{m+mRunningSumClient} application or\\
\examplesNameR{Adapters}{m+mRunningSumAdapter} or
\examplesNameR{Adapters}{m+mRunningSumAltAdapter} adapters; it returns the running sum
associated with the requesting application or adapter.\\

Note that the application will exit if the \serviceNameR[\RS]{RegistryService} is not
running.
\insertAutoAppParameters
\insertTagDescription{Running Sum}
The matching client application does not directly use the endpoint name to establish a
connection to the service.\\

The \requestsNameD{Examples}{Examples}{addtosum} request updates the running sum
corresponding to the requesting application or adapter and returns the resulting value.
Note that an \requestsNameX{Examples}{Examples}{addtosum} request implicitly starts the
calculation.\\
\requestsNameE{Examples}{Examples}{addtosum}%

The \requestsNameD{Examples}{Examples}{resetsum} request clears the running sum
corresponding to the requesting application or adapter.\\
\requestsNameE{Examples}{Examples}{resetsum}%

The \requestsNameD{Examples}{Examples}{startsum} request starts the running sum
calculation for the requesting application or adapter.
Note that a \requestsNameX{Examples}{Examples}{startsum} request implicitly resets the
calculation, if it has already been started.\\
\requestsNameE{Examples}{Examples}{startsum}%

The \requestsNameD{Examples}{Examples}{stopsum} request stops the running sum
calculation for the requesting application or adapter.
\requestsNameE{Examples}{Examples}{stopsum}%
\condPage{}
If the service is selected for execution from within the \emph{\MMMU} application, the
following dialog will be presented:
\objScaledDiagram{mpm_images/launchRunningSumService}%
{launchServiceRunningSum}{Launch options for the \emph{Running Sum} service}{0.8}

\insertTagAndEndpointDescription{mpm_images/runningRunningSumService}%
{serviceRunningRunningSum}{The \emph{\MMMU} entity for the \emph{Running Sum} service}%
{1.0}
\tertiaryEnd{\examplesNameE{Services}{m+mRunningSumService}}
\tertiaryStart{\examplesNameP{Services}{m+mTruncateFloatFilterService}}
The \examplesNameX{Services}{m+mTruncateFloatFilterService} application is a Filter
service, converting a stream of floating\longDash{}point values into a stream of integers.
The application responds to the standard Filter service requests and can be used as a
standalone data transformer, without the need for a client connection.\\

The \requestsNameR{\inputOutput}{InputOutput}{configuration} request has no arguments and
returns nothing.\\

The \requestsNameR{\inputOutput}{InputOutput}{configure} request has no arguments and
does nothing.\\

The \requestsNameR{\inputOutput}{InputOutput}{restartStreams} request stops and then
starts the input and output streams.\\

The \requestsNameR{\inputOutput}{InputOutput}{startStreams} request initiates listening
on the input stream; the output stream is only written to when input is seen, so nothing
special is done with it.\\

The \requestsNameR{\inputOutput}{InputOutput}{stopStreams} request terminates listening
on the input stream, which will cause sending on the output stream to cease as well.\\

Note that the application will exit if the \serviceNameR[\RS]{RegistryService} is not
running.\\

\insertAppParameters
\insertTagDescription{Truncate Float Filter}
\insertFilterServiceComment\\

\insertStandardServiceCommands
\condPage{}
If the service is selected for execution from within the \emph{\MMMU} application, the
following dialog will be presented:
\objScaledDiagram{mpm_images/launchTruncateFloatFilterService}%
{launchServiceTruncateFloat}{Launch options for the \emph{Truncate Float Filter} service}%
{0.8}

\insertTagAndEndpointDescription{mpm_images/runningTruncateFloatFilterService}%
{serviceRunningTruncateFloat}{The \emph{\MMMU} entity for the \emph{Truncate Float Filter}
service}{1.0}
\tertiaryEnd{\examplesNameE{Services}{m+mTruncateFloatFilterService}}
\secondaryEnd
\condPage
\secondaryStart{Example~Clients}
The example client applications provided use simple terminal\longDash{}based interfaces to
generate requests for their matching service applications.
\tertiaryStart{\examplesNameP{Clients}{m+mEchoClient}}
The \examplesNameX{Clients}{m+mEchoClient} application reads input from a terminal and
sends it to the \examplesNameR{Services}{m+mEchoService} application using an
\requestsNameR{Miscellaneous}{Miscellaneous}{echo} request; it exits when an empty line is
entered.\\

\insertShortClientParameters{}

Note that the application will also exit if the \serviceNameR[\RS]{RegistryService} or the
\examplesNameR{Services}{m+mEchoService} application are not running or if the application
is not running from an interactive terminal.
\tertiaryEnd{\examplesNameE{Clients}{m+mEchoClient}}
\tertiaryStart{\examplesNameP{Clients}{m+mRandomNumberClient}}
The \examplesNameX{Clients}{m+mRandomNumberClient} application reads the count of
random numbers desired from a terminal and sends a
\requestsNameR{Examples}{Examples}{random} request to the
\examplesNameR{Services}{m+mRandomNumberService} application; it exits when a zero is
entered for the count of random numbers desired.

\insertShortClientParameters{}

Note that the application will also exit if the \serviceNameR[\RS]{RegistryService} or the
\examplesNameR{Services}{m+mRandomNumberService} application are not running or if the
application is not running from an interactive terminal.
\tertiaryEnd{\examplesNameE{Clients}{m+mRandomNumberClient}}
\condPage
\tertiaryStart{\examplesNameP{Clients}{m+mRunningSumClient}}
The \examplesNameX{Clients}{m+mRunningSumClient} application reads commands from a
terminal and sends corresponding requests to the
\examplesNameR{Services}{m+mRunningSumService} application.\\

The commands are:
\begin{itemize}
\item\cmdItem{?}{display a list of the available commands}
\item\exSp\cmdItem{+}{read a number from the terminal and send it to the service via an
\requestsNameR{Examples}{Examples}{addtosum} request, to update the running sum for this
client}
\item\exSp\cmdItem{q}{send a \requestsNameR{Examples}{Examples}{stopsum} request to the
service so that it will stop calculating the running sum for this client and then exit the
application}
\item\exSp\cmdItem{r}{send a \requestsNameR{Examples}{Examples}{resetsum} request to the
service so that it will reset the running sum for this client}
\item\exSp\cmdItem{s}{send a \requestsNameR{Examples}{Examples}{startsum} request to the
service so that it will start calculating the running sum for this client}
\end{itemize}

\insertShortClientParameters{}

Note that the application will also exit if the \serviceNameR[\RS]{RegistryService} or the
\examplesNameR{Services}{m+mRunningSumService} application are not running or if the
application is not running from an interactive terminal.
\tertiaryEnd{\examplesNameE{Clients}{m+mRunningSumClient}}
\secondaryEnd
\condPage
\secondaryStart{Example~Adapters}
Adapters are applications that use a client object to connect to an \mplusm{} service,
routing input from one or more input \yarp{} network connection via the client object to
the service and returning results from the client to one or more output \yarp{} network
connections.\\

The example adapters demonstrate using a single input \yarp{} network connection
[\examplesNameR{Adapters}{m+mRandomNumberAdapter} and
\examplesNameR{Adapters}{m+mRunningSumAltAdapter}] or multiple input \yarp{} network
connections [\examplesNameR{Adapters}{m+mRunningSumAdapter}].\\

Adapters have ten optional parameters:
\begin{itemize}
\item\optItem{a}{}{args}{display the argument descriptions for the executable and leave
\longDash{} note that this option is primarily for use by the \emph{\MMMU} application}
\item\exSp\optItem{c}{}{channel}{display the endpoint name after applying all other
options and leave}
\item\exSp\optItem{e}{v}{endpoint}{specifies an alternative endpoint name `\textit{v}' to
be used}
\item\exSp\optItem{g}{}{go}{indicates that the adapter is to be started immediately}
\item\exSp\optItem{h}{}{help}{display the list of optional parameters and arguments and
leave}
\item\exSp\optItem{i}{}{info}{display the type of the executable, the valid options, the
matching criteria used to locate the service to which the adapter will attach and a
description of the executable and leave \longDash{} note that this option is primarily
for use by the \emph{\MMMU} application}
\item\exSp\optItem{p}{v}{port}{specifies the port number `\textit{v}' to be used, if a
non\longDash{}default port is desired}
\item\exSp\optItem{r}{}{report}{report the adapter metrics when the application exits}
\item\exSp\optItem{t}{v}{tag}{specifies the tag `\textit{v}' to be used as part of the
adapter name}
\item\exSp\optItem{v}{}{vers}{display the version and copyright information and leave}
\end{itemize}

Note that adapters normally have no direct interface.
That is, they are `faceless' applications and perform their operations without user
interaction.
\condPage
\tertiaryStart{\examplesNameP{Adapters}{m+mRandomNumberAdapter}}
The \examplesNameX{Adapters}{m+mRandomNumberAdapter} application reads the count of
random numbers desired from an input \yarp{} network connection and sends a
\requestsNameR{Examples}{Examples}{random} request to the
\examplesNameR{Services}{m+mRandomNumberService} application.
The random numbers returned from the service are sent out an output \yarp{} network
connection.
Note that the application will also exit if the \serviceNameR[\RS]{RegistryService} or the
\examplesNameR{Services}{m+mRandomNumberService} application are not running.\\

\insertStandardAdapterCommands{}
If the adapter is selected for execution from within the \emph{\MMMU} application, the
following dialog will be presented:
\objScaledDiagram{mpm_images/launchRandomNumberAdapter}%
{launchAdapterRandomNumber}{Launch options for the \emph{Random Number} adapter}{0.8}
\condPage
\insertTagAndEndpointDescription{mpm_images/runningRandomNumberAdapter}%
{adapterRunningRandomNumber}{The \emph{\MMMU} entity for the \emph{Random Number}
adapter}{1.0}
\tertiaryEnd{\examplesNameE{Adapters}{m+mRandomNumberAdapter}}
\tertiaryStart{\examplesNameP{Adapters}{m+mRunningSumAdapter}}

The \examplesNameX{Adapters}{m+mRunningSumAdapter} application reads  values from an input
`data' \yarp{} network connection and commands from an input `control' \yarp{} network
connection and sends corresponding requests to the
\examplesNameR{Services}{m+mRunningSumService} application.\\

If a numeric value is read from the `data' input \yarp{} network connection, it is sent to
the service via an \requestsNameR{Examples}{Examples}{addtosum} request, to update the
running sum for this adapter.
If a string is read from the `control' input \yarp{} network connection, it is considered
to be one of the following commands:
\begin{itemize}
\item\textbf{q:} exit the application.
\item\exSp\textbf{r:} send a \requestsNameR{Examples}{Examples}{resetsum} request to the
service so that it will reset the running sum for this adapter.
\item\exSp\textbf{s:} send a \requestsNameR{Examples}{Examples}{startsum} request to the
service so that it will start calculating the running sum for this adapter.
\item\exSp\textbf{x:} send a \requestsNameR{Examples}{Examples}{stopsum} request to the
service so that it will stop calculating the running sum for this adapter.
\end{itemize}
The running sums returned from the service are sent out an output \yarp{} network
connection.\\

Note that the application will also exit if the \serviceNameR[\RS]{RegistryService} or the
\examplesNameR{Services}{m+mRunningSumService} application are not running.\\

\insertStandardAdapterCommands
\condPage{}
If the adapter is selected for execution from within the \emph{\MMMU} application, the
following dialog will be presented:
\objScaledDiagram{mpm_images/launchRunningSumAdapter}%
{launchAdapterRunningSum}{Launch options for the \emph{Running Sum} adapter}{0.8}

\insertTagAndEndpointDescription{mpm_images/runningRunningSumAdapter}%
{adapterRunningRunningSum}{The \emph{\MMMU} entity for the \emph{Running Sum} adapter}%
{1.0}
\tertiaryEnd{\examplesNameE{Adapters}{m+mRunningSumAdapter}}
\condPage
\tertiaryStart{\examplesNameP{Adapters}{m+mRunningSumAltAdapter}}
The \examplesNameX{Adapters}{m+mRunningSumAltAdapter} application reads  values and
commands from an input \yarp{} network connection and sends corresponding requests to the
\examplesNameR{Services}{m+mRunningSumService} application.\\

If a numeric value is read from the input \yarp{} network connection, it is sent to the
service via an \requestsNameR{Examples}{Examples}{addtosum} request, to update the running
sum for this adapter.
If a string is read from the input \yarp{} network connection, is is considered to be one
of the following commands:
\begin{itemize}
\item\textbf{q:} exit the application.
\item\exSp\textbf{r:} send a \requestsNameR{Examples}{Examples}{resetsum} request to the
service so that it will reset the running sum for this adapter.
\item\exSp\textbf{s:} send a \requestsNameR{Examples}{Examples}{startsum} request to the
service so that it will start calculating the running sum for this adapter.
\item\exSp\textbf{x:} send a \requestsNameR{Examples}{Examples}{stopsum} request to the
service so that it will stop calculating the running sum for this adapter.
\end{itemize}
The running sums returned from the service are sent out an output \yarp{} network
connection.\\

Note that the application will also exit if the \serviceNameR[\RS]{RegistryService} or the
\examplesNameR{Services}{m+mRunningSumService} application are not running.\\

\insertStandardAdapterCommands
\condPage{}
If the adapter is selected for execution from within the \emph{\MMMU} application, the
following dialog will be presented:
\objScaledDiagram{mpm_images/launchRunningSumAltAdapter}%
{launchAdapterRunningSumAlt}{Launch options for the \emph{Running Sum} alternate adapter}%
{0.8}

\insertTagAndEndpointDescription{mpm_images/runningRunningSumAltAdapter}%
{adapterRunningRunningSumAlt}{The \emph{\MMMU} entity for the \emph{Running Sum} alternate
adapter}{1.0}
\tertiaryEnd{\examplesNameE{Adapters}{m+mRunningSumAltAdapter}}
\secondaryEnd
\appendixEnd{}
